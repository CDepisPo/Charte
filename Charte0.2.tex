\documentclass[a4paper,12pt]{article}

\usepackage[utf8]{inputenc}
\usepackage[T1]{fontenc}      % un second package
\usepackage[francais]{babel}
\usepackage[a4paper, left=3cm, right=3cm, top=3cm, bottom=3cm]{geometry}
\usepackage{default}

% \makeatletter
% \renewcommand\chapter{%
%                     \thispagestyle{plain}%
%                     \global\@topnum\z@
%                     \@afterindentfalse
%                     \secdef\@chapter\@schapter}
% \makeatother



%opening
\title{\Huge{La Charte du club débat d'épistémologie politique} \LARGE{V0.2.0}}
\author{Voté à la majorité simple ou supérieure par l'ensemble des membres de tous les clubs}
\date{01/04/2016}

\begin{document}

\maketitle
\tableofcontents
\newpage

%%%%%%%%%%%%%%%%%%%%%%%%%%%%%%%% Les principes directeurs des clubs %%%%%%%%%%%%%%%%%%%%%%%%%%%%%%%%
\part{Les principes directeurs des clubs}
\paragraph{Introduction}
Ce club est ouvert à tous les citoyens et futurs citoyens qui cherchent à nourrir leur réflexion de l'opinion des autres. Ceux qui cherchent, au delà des frontières, des apparences, un monde où les hommes pourraient cohabiter d'une manière plus harmonieuse. A tous ceux ci, le club ne leur est pas seulement ouvert, il a été fait pour eux, par des gens qui, comme tant d'autre, partagent cet idéal simple. Si vous êtes à la recherche d'autres personnes qui aiment réfléchir à comment bien faire bouger les choses, si vous aimez prendre le temps d'explorer le monde des idées au delà des préjugés et des apparences, nous sommes à la recherche de personnes telles que vous. En espérant que cette charte vous inspirera, bonne lecture.

\section{A quoi sert cette charte?}

\paragraph{Qui controle la charte?}
La charte est faite pour être controlée d'une manière démocratique qui fonde sa légitimité. Pour plus d'information, renseignez vous dans la section explicitant le fonctionnement du site web.
\subparagraph{Légitimité de la charte}
Vous vous demandez peut-être ce qui fait que la charte mérite d'être suivie à la lettre. Sachez d'abord que si la présente partie de la charte, communément nommée tête de charte est la partie la plus solide et la plus inamovible, le reste de la charte est modulaire et soumis à un débat permanent entre les membres et les clubs. Chaque club ayant délibérément choisis de se ranger derrière la dénomination d' "Antenne du club débat d'épistémologie politique", il a explicitement accepté de se soumettre aux principes de bases édictés dans la première partie de la présente charte. Ayant librement décidé d'adhérer à cette première partie, elle est donc légitime à vos yeux, et vous vous devez donc de la respecter sans autres réserves que celle qu'elle admet elle même,car vous êtes affiliés au club. On peut se dire que la tête de la charte est l'équivalent d'une constitution, relativement stable comparé aux lois, qui vont en dessous. Si vous n'êtes pas d'accord avec certains points, vous pouvez les rendres plus légitimes en les remettant en cause sur le site du club grâce à une procédure démocratique détaillée dans la partie explicitant le fonctionnement du site.

\paragraph{Lire et interpréter la charte}
Pour un plus grand confort de lecture, la charte est écrite en plusieurs parties indépendantes, qui vous permettent de vous y retrouver facilement. Après les principes éthiques, tout ce qui se trouve dans la charte cesse de former un ensemble inamovible (bien que toute la charte soit modifiable, j'y reviendrai). C'est à dire qu'à la création de votre club, il vous est demandé de faire des choix entre les différentes possibilités de fonctionnements pour votre club (se référer à la section qui parle de la création d'un nouveau club). Pour ce qui est de l'interprétation, si vous trouvez des conflits entre le contenu de la charte et sa "tête", merci d'en faire part sur le site. Pour régler toute incohérence ou marge d'interprétation, référez vous à la tête de la charte, sa constitution et agissez conformément à ses principes de base. En cas de désaccord, c'est le doyen de séance (ou << Oeil >>, voir partie dédiée) qui décide de la marche à suivre, mais n'oubliez pas de faire part sur le site de tout problème rencontré avec la charte.

\paragraph{Origine du nom}
Au premiers temps du club, le nom temporaire de "club de réflexion citoyenne" fut utilisé. Cependant, il devint évident que ce nom n'était pas suffisamment représentatif de notre projet. Nous cherchions un nom qui soit plus en rapport avec les valeurs de ce club, un nom qui incarnerait notre souhait d'une réflexion plus globale sur notre monde, portée sur la politique et l'aspect analytique, méthodologique et rationnel que nous comptions porter sur nos opinions. Après plusieurs débats, nous adoptâmes finalement l'épistémologie, c'est à dire l'étude critique des sciences, appliquée à la politique, avec au premier plan la notion de débat. A ce jour, ce nom est l'interprétation la plus consensuelle des principes de notre club.
  
\section{Principes Ethiques}
\paragraph{} 
Pour se préserver des préjugés et des raisonnements binaires qui enrayent tant de débats. Pour fournir au curieux un lieu qui lui permetterait d'échanger les opinions qu'il avaient jusqu'alors du mal à partager. Enfin, pour donner matière à réfléchir sur le monde et sur les autres, voici ci-dessous les raison d'être de ce club. Parce que le dialogue est la condition indispensable à la paix sociale et à la démocratie véritable, c'est sur ce principe que se fonde notre démarche. La suite de la tête de la charte couche notre éthique, pour que notre méthode se propage dans des cadres toujours plus larges.

\paragraph{Axiomes}
Nous avons établis, dans un premier temps tacitement, différends "axiomes". Ce sont les faisceaux directeurs de notre éthique et de nos réflexion, et s'ils nous semble relever du bon sens, leur remise en cause reste un sujet discutable au sein des clubs (comme nimporte quel autre). La création et le développement de cette charte a été influencé notemment par la zététique et les divers courants anarchistes.

\paragraph{Les axiomes du club:}
\begin{itemize}
 \item  Le but du club est principalement pédagogique, il cherche à apprendre aux élèves à se défendre contre les argumentations fallacieuses.
 \item  Pour permettre le consensus, nos réflexions partent du présuposé qu'elles doivent maximiser à long terme le bien être de la majorité.
 \item  Nous essayons de tendre grâce au débat vers une vision plus objective et universelle du monde (qui parle à tous les hommes, sans distinction de culture ou de croyance).
 \item  Le club est laïque. Bien que respectant les croyances de chacun, nos idées sont, autant que possible, portés par la logique et la méthode scientifique (à ne pas confondre avec le scientisme).
 \item  Plutôt que la Vérité, nous cherchons l'option la plus raisonnable compte tenu des éléments en notre possession, tout en gardant sous la main les autres options possibles si jamais des faits nouveaux nous dirigeaient dans leur sens.
\end{itemize}

\section{Règles de bases du débat}

\paragraph{Les règles de base}
Voici quelques règles que vous devez intégrer lorsque vous entrez dans l'enceinte du club. Il est important de les appliquer sincèrement pour non seulement ne pas devenir un obstacle au débat rationnel et paisible, mais pour en se mettre en disposition de pouvoir offrir aux autres le meilleur de vous même:
\begin{description}
 \item[Règle 1 :] En entrant dans le débat, vous êtes prêts à considérer sans violence tous les arguments.
 \item[Règle 2 :] Vous cherchez à connaître clairement vos propres opinions, leurs arguments et leurs limites, pour rester humble et être prêt à en changer.
 \item[Règle 3 :] Vous vous efforcez de tendre vers l'objectivité dans l'analyse de vos propres arguments et de ceux des autres membres.
 \item[Règle 4 :] De part sa vocation et son éthique, le club de réflexion citoyenne ne peut tolérer la violence sous toute ses formes, vous vous engagez à la dénoncer et à garder un certain détachement vis à vis de vos émotions puissantes.
\end{description}
Pour vous guider dans l'application de ces rêgles, elles sont appuyés par les documents en annexe, dont la lecture est conseillée, car ils fournissent un point de départ et des ouvertures à qui veut se former à l'art de l'argumentation et de la défense réthorique. 

\paragraph{Manque de respect à la charte}
L'obéissance à la charte permet de lier tous les clubs et de mettre au clair les principes de base du débat démocratique. Si un membre agit à l'encontre de la charte, et qu'il refuse de le respecter malgrès qu'on lui ait fait remarquer son erreur, la charte propose quelques conseils pour pouvoir poursuivre le débat en le mettant à l'écart, proportionellement à sa faute:
\begin{itemize}
  \item Si la faute est assez faible et que l'auteur l'assume plus ou moins, par exemple lors d'un emportement ou qu'il n'a de cesse d'interrompre ses camarades, s'il a du mal à se calmer seul tout en continuant le débat, il doit se mettre à l'écart pour se remettre en question, se calmer, et reprendre le débat dans de meilleures conditions.
  \item Si le membre commet des fautes répétées et qu'il refuse de se remettre en question et de respecter la charte, il devient un poids pour le club. Il faut alors en discuter avec lui. Si malgrès cela il ne veux rien entendre, vous pouvez l'empêcher de parler pendant 30 minutes, ou bien l'exclure jusqu'à qu'il se décide à respecter les autres membres, selon sa personnalité et ses dispositions à re remettre dans de bonnes conditions que l'oeil devra juger.
  \item En cas de faute très grave (insultes à répétition, menaces, etc...), le membre doit être exclu de la séance par l'oeil, voir même renvoyé du club à la majorité simple.
  \item Pour éviter que ce ne soit pire, je vous conseille de faire attention à qui vous parrainez, sinon (les débats peuvent être éprouvants) le reste risque de devenir du ressort du personnel de votre établissement, de la police, ou même de l'armée.
\end{itemize}


%%%%%%%%%%%%%%%%%%%%%%%%%%%%%%%% Rôles et organisation humaine %%%%%%%%%%%%%%%%%%%%%%%%%%%%%%%%
\part{Rôles et organisation humaine}
\section{L'Oeil et le Scribe}
Les modes d'élection de l'Oeil et du Scribe est décidé dans la charte locale.
\paragraph{L'Oeil}
L'oeil est le modérateur de la séance. Son mode d'élection est décidé dans la charte locale. Il a le rôle de maintenir les membres dans le sujets et de nourrir une réflexion construite. Selon le type de séance, il pourra par exemple désigner qui a la parole, l'orientation du sujet en cours de séance, interrompre les hors sujets abusifs ou arrêter le débat quand il ne le juge plus pertinent (il doit bien évidemment justifier son interruption). Certains débats, comme les débats confidentiels, peuvent se passer d'oeil. Pour l'aider dans sa tâche, le site web propose des documents vidéos et des conseils, et l'annexe, que l'aspirant Oeil doit avoir lu absolment contiens le minimum vital pour pouvoir discerner les remarques pertinentes de celles qui le sont moins. La bonne connaissance de la réthorique et une organisation rigoureuse est primordiale pour faire un bon oeil. 

\paragraph{Le Scribe}
Le scribe est le secrétaire de la séance. Son rôle n'est pas facile car il doit suivre le débat tout en le transcrivant, éventuellement en y participant dans les limites de ses capable. Il a la tâche de transcrire les arguments, leurs points forts et la manière dont les membres les ont reliés entre eux pour offrir à l'archiviste une fiche à partir de laquelle il pourra écrire les conclusions de séance et aux débats futurs qui se nourrirons des anciens au lieu de se répéter. Le moment venu, on ne regrette pas d'avoir gardé une trace de ce qui a été dis. Les membres peuvent faire part au scribe de suggestions sur ce qui devrait être noté, au cas où il n'aurrait pas assez insisté dessus où les aurait fait des erreurs et oublis. Le scribe doit prendre note de toutes les réflexions produites pendant la séance, ce qui implique une vitesse de frappe correcte au clavier. Il peut pour celà s'aider de toutes sortes d'outils, ordinateur, appareil photos, tablettes graphiques, feuilles, etc... Tant que l'Archiviste pourra y accéder sans difficultés. C'est également lui que l'Archiviste appellera en priorité si jamais il y a un point des notes qu'il a prise qui demandent clarification.

\section{A propos du recrutement de nouveaux membres}
\paragraph{Le parrainage}
Un des systèmes possible est le système du parrainage pour recruter de nouveaux membres. Il a l'aventage de permettre d'être sûr de garder le contrôle sur les nouveaux venus et qu'ils bénéficieraient à leur entrée du soutiens d'au moins un membre. Ce système suppose qu'une personne trouve un des membres du club qui accepte de devenir son Parrain, et lui même son Aspirant. L'ombre doit ensuite participer à deux séances pour pouvoir être éligible au status de membre, à la troisième séance, un vote est effectué. L'Aspirant et son Parrain exposent les motivations de l'Aspirant, puis il faut qu'au mois 80\% des membres soient d'accord sa venue, auquel cas l'Aspirant pourra choisir de devenir membre ou membre à responsabilité.
\subparagraph{En cas d'opposition}
Classiquement, le vote commence à main levé en demandant qui s'oppose à la venue du nouveau membre. Les membres s'opposants à l'élection de l'ombre sont tenus de se justifier, peut s'en suivre un bref débat (-4 min) à la suite duquel est fait un vote, procédant de la manière choisie dans la charte locale. Si l'assemblée des membres refuse l'Ombre, celui ci restera ombre jusqu'à être élu à une séance ultérieure, s'être lui même désisté, ou que son parrainage ait finalement cessé.

\paragraph{Différend types de scrutins}
Il y a heureusement ensuite de nombreuses manières de choisir ces différends membres, toutes sont valables et ont leurs propres arguments, laissez moi vous présenter celles que nous avons utilisés ou imaginés au club du lycée Monge:
\begin{description}
 \item[L'ordre prédéfinit :] Les noms des membres sont notés sur une feuille, puis ils sont tirés successivement en fonction de leur présence/absence.
 \item[Variante et règles supplémentaires :] Les membres peuvent être tirés au hasard puis choisis dans l'ordre ainsi généré, ou bien inversés à chaque fois. En cas d'absence d'un membre, son nom est noté comme prioritaire et il sera automaiquement élu la prochaine absence. Sauf s'il est absent régulièrement                                     , auquel cas il faut le mettre dans une colonne séparée pour qu'il ne soit élu que durant ces séances pour ne pas être élu quasiment deux fois plus que les autres. Je vous laisse compléxifier les règles à votre fantaisie.
 \item[Le tirage au sort :] Le tirage au sort nous a tout de suite semblé le moyen le plus raisonnable de choisir les membres, car il les met statistiquement tous à égalité. Il est précisé dans la section précédente quelques moyens de procéder pour le tirage au sort.
 \item[Variante et règles supplémentaires :] Mais les statistiques sont perturbées lorsque un membre s'absente. S'il est absent souvent, il paricipera moins aux tâches de directon du club. Un moyen de pallier à ce problème est d'attribuer à chaque membre à responsabilité un nombre de point. celui-ci croit de 1 à chaque séance où il est présent, de 2 lorsqu'il est absent et est remis à 0 lorqu'il est sélectionné. Chaque point compte alors comme une chance d'être élu, et celui qui en possède le plus est par conséquent plus prompt à être élu.
 \item[L'élection :] Les membre à responsabilité peuvent aussi être élus, soit par tous les membres soit entre membres à responsibilité. Nous n'avons pas choisis ce système au lycée Monge car il créait de la concurrence là où nous jugions que chacun devait participer pour le confort de débat de tous. Mais essayez, vous nous ferez part de vos retours!
\end{description}

\paragraph{Qui peut être élu?}
Une question intéressante, car au lycée Monge, nous avons séparés les membres officiels en deux catégories: ceux qui acceptaient de posséder des rôles dans le club, et les autres. cependant, je vous enjoins vivement à tester d'autres manières de faire car l'état actuel des choses ne satisfait pas tous nos membres. Quelques pistes pour vos membres à responsabilités à tester:
\begin{itemize}
 \item Tout membre est par défaut un membrte à responsabilité [Ju] Il faut accepter tous les travaux, par solidarité
 \item Les membres peuvent choisir un ou plusieurs rôles (demande certainement des rêges supplémentaires pour rester juste)
\end{itemize}

Pour les différentes méthodes d'élection, voir les conventions plus bas.

\section{L'Archiviste}
\paragraph{Rôle de l'Archiviste}
L'Archiviste fait office de gestionnaire du club. C'est lui qui fait en sorte que tout soit en ordre, qui compose les conclusions de séances pour les soumettre à l'approbation des autres membres et aboutir à un projet, il est également responsable de la publication des articles de son club sur le site du club qu'il doit relire et tiens à jour le répertoire commun du club ainsi que les documents relatifs à son organisation. En somme, l'Archiviste est le pilier du club, celui qui travaille en backstage pour que le club fonctionne sur le long terme.
\subparagraph{A propos des conclusions de séance}
L'Archiviste est en effet chargé de composer, entre deux séance, un document suivant une organisation très stricte et hiérarchique comme décrite dans la partie portant sur l'organisation d'un texte de conclusion. Il est basé sur un dialogue entre l'archiviste et le scribe de fonction lors de la séance ainsi que sur les notes prises par celui-ci. Le but d'une conclusion est de fournir aux membres un aperçu synthétique du fond de ce qui a été débattu pour les mettre faces à l'avancement actuel du débat, sur les questionnements qui auraient pu être oubliés ou sur les questions qui n'avaient pas été assez développés pour permettre de décider sur quel points prioritaires se concentrera le débat.

\paragraph{Désignation de l'Archiviste}
Au lycée Monge, nous procédont comme suit, mais libre à vous encore une fois de prendre une autre méthode de désigation. L'archiviste est désigné toutes les trois semaines par élection de tous les membres, mais vous pouvez piquer d'autres idées de désignations dans la partie sur l'oeil et le scribe. 

\section{Organisations de la parole}
\paragraph{Différentes organisations}
Il est avisé de changer l'organisation du débat en fonction du nombre de membres présents. Pour celà, nous allons séparer les séances en 5 catégories en fonction du nombre de membres:
\begin{description}
 \item[2-4 membres :] Débat confidentiel
 \item[5-10 membres :] Groupe de réflexion
 \item[11-25 membres :] Petite assemblée
 \item[26-60 membres :] Assemblée
 \item[61+ membres :] Grande assemblée
\end{description}

Il est généralement avisé d'éviter de débattre directement une fois atteint un certain nombre de membres, car il est admis que le débat est humainement impossible avec autant de personnes à la fois (l'expérience montre que les membres tendent à se rassembler de toute façon en unités de réflexion plus petites au sein de cette assemblée). La compartimentation du club peut se faire à partir du statut de petite assemblée, au dessus, elle est plus que conseillée. Ci-dessous sont développés des conseils sur les organisations et conventions à adopter en fonction du nombre de membres que vous acceuillez.

Note: Dans tous les cas, si une séance contient la majorité supérieure des membres présents, l'Archiviste devra soumettre ses conclusions en début de séance. S'il est impossible de réunir tous les membres (ce qui est une situation compliquée, mais malheureusement possible) faites voter tous les membres (jusqu'à atteinte de la majorité supérieure) sur plusieurs séances, et comptabilisez simplement le total. Je précise cependant que le club du lycée Monge n'ayant pas eu à faire face à ce genre de problème, c'est à vous qu'il revient d'ajouter de nouvelles règles à cette manoeuvre pour la rendre plus juste, d'aucun diraient plus logique.

\paragraph{Débat confidentiel}
Lorsque vous êtes moins de cinq, les règles du débat importent peu: en effet, c'est rarement lors de ce genre de séances que l'on imposera des changements importants pour le club. En revanche, il est courrant d'en profiter pour en discuter dans ce cadre plus intime. Par exemple, au club du lycée Monge, nous utilisons ces séances pour parler de sujets qui nous tiennent à coeur mais qui sont parfois hors de la problématique, pour se changer les idées. Ou alors, nous débattons sur des points de la charte. Plus mûres, nous pourrons ains proposer des idées plus construite aux autres membres du club pendant une séance plus fréquentée. Il est conseillé d'élire un Scribe (voir la section afférente), en revanche la présence d'un oeil est plus encombrante qu'autre chose. Le système de prise de parole spontanée est le plus commun.

\paragraph{Groupe de réflexion}
C'est là que le club devient réellement opérationnel. Pour l'organisations, jetez un coup d'oeil à la section nommée << Déroulement d'une séance de débat >>. Il est conseillé d'utiliser une organisation de type oeil et scribe pour ce genre de débat. 

\paragraph{Petite assemblée}
Bien qu'il soit toujours possible de tenir un débat classique, le grand nombre d'intervenants potentiel ralentira le débat et en laissera certains qui ne veulent pas le ralentir de coté. Il est donc conseillé dans ce cas de fractionner le club en deux, puis de mettre en commun les conclusions pour les débattre à nouveaux. Cette manière de faire permet de traiter de deux sujets à la fois, ou bien de faire une séance recherche en même temps qu'une séance débat. N'oubliez pas cependant que les conclusions doivent être votés à la majorité représentative et que la majorité supérieure des membres doit être présente. Il est cependant, notamment durant les séances spéciales invités possible de garder le club uni, car le débat est recentré autour d'une seule personne.

\paragraph{Assemblée}
Si vous possédez autant de membres dans votre club, félicitations. S'ils peuvent venir, tant mieux. Je conseille en cas de séance débat et/ou recherche de partitionner le club en 3 à 7 groupes de réflexion.
Seuls les votes de conclusions ainsi que les élections de nouveaux membres se feront avec la totalité des membres.

\paragraph{Grande Assemblée}
La grande Assemblée suit par défaut les mêmes règles que l'Assemblée, le club de réflexion citoyenne n'ayant encore jamais eu à traiter un tel nombre de membres présents simultanéménts. 

\section{Hiérarchie des membres}
Voici la liste des différents grades possibles des membres, voir les sections dédiées pour plus d'information.
\begin{description}
 \item[Membres à responsabilité] C'est un membre qui accepte d'être éligible aux fonction de scribe et d'oeil.
 \item[Membre] Le membre a droit de prticiper aux votes, mais n'est pas éligibles aux tâches d'oeil ou de scribe.
 \item[Ombre] L'aspirant est parrainé par un membre et apsire à en devenir moyennant deux séances d'observation. Durant celles-ci, il peut participer au débat, mais pas aux votes.
 \item[Invité] Une personne extérieur a club qui est invité temporairement de part son status d'expert ou plus généralement son utilité dans le cadre d'un débat précis. L'invité est là pour débattre et doit se plier aux règle de débat établis, mais ne participe pas aux votes.
 \item[Spectateur] Peut observer, poser des questions quand autorisé, mais sans plus. Nimporte qui, pour peut qu'il ne dérange pas trop le débat peut devenir spectateur.
\end{description}



%%%%%%%%%%%%%%%%%%%%%%%%%%%%%%%% Organisation des Séances %%%%%%%%%%%%%%%%%%%%%%%%%%%%%%%%
\part{Organisation des séances}

\section{Chartes locales}
Les clubs doivent décider lors de leur première séance à de la création d'une charte locale, contenant les modalités des règles adoptés par cette antenne du club. La charte se compose de la première tête de la charte officielle, inchangée, suivie des modalités adoptés. Une modalité d'écrit comme suit:
[Nom du point soumis à modalité] : [Nom de la modalité choisie]; [Précisions concernant les paramètres de la modalité, s'enchainent par des virgules]
exemple: Elections de l'Oeil : Ordre prédéfini; dans l'ordre alphabétique des noms de famille

\paragraph{}
Ci dessous la liste des points de cette charte soumis aux modalités:
\begin{description}
 \item [Mode de recrutement des nouveaux membres]
 \item [Mode d'élection de l'Oeil]
 \item [Mode d'élection du Scribe]
 \item [Mode d'élection de l'Archiviste]
 \item [Durée d'élection de l'Archiviste]
 \item [Organisation des membres au sein du club]
\end{description}


\section{Précisions sur les relations interpersonnes}
Puisque le meilleur moyen d'atteindre un idéal est de commençer à l'appliquer dès aujourd'hui, certains comportements doivent être respectés entre deux individus qui décident de se parler en tant que membre de CDEpisPo, que ce soit à l'intérieur ou à l'extérieur:
\paragraph{Respect}
La hiérarchie n'est pas une échelle de valeurs, seulement de responsabilité. Il en va de même pour l'âge. Ainsi, jeune ou moins jeune, Archiviste comme membre sans responsabilités sont absolument égaux sur le plan moral. Le fait d'insister sur l'âge ou la position sociale peu importe la raison n'est pas acceptable au sein du club. De fait, il est capital de considérer toujours son prochain comme son égal dans le dialogue. Ce principe est de rigeur entre les membres pendant les séances. S'il peut l'être parfois aussi au dehors, tant mieux.

\section{Déroulement d'une séance de débat}
Ci dessous le shéma type d'une séance de débat normale:
\begin{enumerate}
 \item Phase d'installation.
 \item Temps où tous les membres peuvent proposer des idées sur nimporte quel sujet.
 \item On procède à l'élection de l'Oeil et du Scribe. L'Oeil est responsable du bon déroulement de la suite de la séance.
 \item S'il y a lieu à une conclusion, l'Archiviste présente son document et fais relire, le soumettant à l'approbation et la critique.
 \item L'oeil récapitule, éventuellement à l'aide d'autres membres l'avançée du projet de réflexion, puis annonce le sujet qui avait été décidé et donne la parole aux premiers intervenants qui vont ouvrir le débat.
 Normalement, cette organisation ne devrait pas dépasser les 30 minutes, et dure moins dans les clubs plus expérimentés.
 \item Le débat s'écoule, et s'il dure plusieures heures, l'oeil et/ou le scribe peuvent décider de procéder à un nouveau tirage au sort pour pouvoir débattre en tant que membres ordinaires, et effectuer un roulement à raison d'un roulement par heure
 \item A la fin de la séance, c'est à dire au moins dix minutes avant la fin, le scribe récapitule l'avancement du projet de réflexion durant cette séance. 
 \item Est ensuite débbatut puis déterminé le ou les sujet du débat de la semaine prochaine, conformément à l'avançée de la thématique du projet de réflexion.
 \item Enfin, chacun peut proposer ses avis sur cette séance et ses idées pour l'améliorer.
 \item Fin de séance, on range la salle, la séance est clos.
\end{enumerate}

\section{Déroulement des différents types de séances pour traiter de différents sujets}
Nous énumérons en effet X types de séances qui desservent les différentes étapes possibles du développement d'un sujet:
\begin{description}
 \item [Les séances de débat :] C'est de ces séances que traite principalement la présente charte, et c'est à ce moment que toutes les réflexions séparées sur un sujet seront rassemblées pour formuler des conclusions et avançer dans la réalisation d'un univers social plus juste.
 \item [Les séances de recherche :] Durant ces séances spéciales, les membres se retrouverons dans une salle informatique si possible dédiée pour pouvoir faire des recherches ensemble. Ils s'organiserons pour trouver des arguments, des réponses sourcées et précises à certaines de leurs interrogations ainsi que des pistes de réflexions en vue de préparer une séance future. Elles se déroulent comme des séances classiques, mais sans scribe (chacun étant considéré comme son propre scribe). Le choix d'un oeil n'est pas obligatoire mais peut dans certains cas aider à s'en tenir à l'objectif originel de la séance. La communication et l'organisation sont des facteurs de réussite décisifs d'une bonne séance de recherche.
 \item [Les séances conférences du club :] Parfois, le club peut avoir à se présenter ou à animer une conférence portant sur certaines de ses conclusions. Celà dépend du type de séance qui sera montré. Typiquement, il peut s'agir d'une présentation suivi d'une séance classique publique. Dans ce cas, l'organisation peut s'apparenter à une séance classique, mais au final vous devrez juger au cas par cas selon l'animation que vous planifiez. Il doit alors s'organiser pour trouver des moyens de communication (diapositives, démonstration, exercices logiques) qui inciteront l'auditoire à la curiosité pour permettre de partager l'esprit du club et créer l'espace de quelques heures un ilot de réflexion citoyenne dans un monde de fous. 
 \item [Les séances spéciales invités :] Si le club invite parfois des personnes externes pour entrainer la discussion plus loin, grâce par exemple à la présence de personnes directement concernées par le sujet, il arrive que la personnalité soit si importante ou primordiale au débat qu'une variante de la séance de débat habituelle est ouverte. Elle met au centre de la discussion l'Invité, qui n'est plus alors soumis aux règles concernant les limites du temps de parole ou les interruptions. La présence d'un oeil dans ce type de séance n'est pas obligatoire. En revanche, on imagine mal une séance aussi importante dépourvue de scribe. La configuration de la salle peut être modifiée pour l'invité (estrade, chaises, hémicycle pour mieux le voir et l'entendre...).
\end{description}

\section{Mener à bien un projet de réflexion}
Pour s'exercer, le club structure sa pensée dans un ou plusieurs projets de réflexion annuel.

Un projet de réflexion représente un sujet général assez vaste qui servira de fil directeur au fil de l'année. Le but du projet est d'être assez fédérateur pour motiver la majorité des membres et général pour pouvoir faire intervenir tous les domaines de l'épistémologie.

Ensuite, une fois le projet de réflexon débatut et voté, deux débuts de séances sont dédiés au choix d'un certain nombre de sujets qui découlent du sujet principal. Une estimation du temps de traitement en heure de chaque sujet doit être adjointe à leur consignation, et il est important d'en proposer un grand nombre.

Au début d'un nouveau projet de réflexion, le début d'une séance (après l'élection de l'oeil et du scribe) est pris pour débattre du nouveau projet à choisir. Premièrement, les membres vont se concerter entre eux et noter sur papier leurs premières idées, puis supprimer celles qu'ils jugent impertientes en regard de la problématique précédement étudiée. Ensuite, chaque membre sera invité à donner ses problématiques sur le tableau, puis on procèdera à un vote en deux tours.
Au premier tour, chaque participant possède la racine carrée du nombre total de proposion arrondie au nombre le plus proche, et vote pour des propositions différentes. S'il y a égalité, un second tour est réalisé, mais cette fois avec un seul vote par personne, ou alors de la même manière que le premier tour. S'il y a encore égalité, on tire au hasard une des possibilités. 
Enfin, il y a un débat concernant les différents sous sujets de la problématique, puis on décide l'ordre des prochaines séances et leur type, avant de commencer la première séance de ce projet.

A la fin d'un sujet, une séance spéciale de conclusion remet en perspective le travail fournit et la valeur des conclusions, définit les modifications apportées à la charte et le degré de réussite des débats. C'est l'occasion de revenir sur les points qui ont bien marchés et sur ceux qui ont empéché le bon débat. Une fois la discution sur ce projet close, on peut soit faire une séance un peut plus abstraite, de philosophe par exemple sur un point qui a laissé les membres pensifs, ou bien commencer un nouveau cycle.

\section{Choix des sujets à traiter}
C'est un choix primordial, puisque le débat constitue le coeur du club. Je liste ici les différentes catégories de sujets qui sont de nature à être traités, conformément à la définition du nom du club, l'épistémologie. Car, en effet, le club traite de la science en politique, ce qui signifie qu'il essaye d'envisager tout ce qui y touche de près ou de loin de la manière la plus logique, donc universelle possible. De fait, lorsque vous comptez débattre de sujets, il est bon de les classer en catégories, et d'organiser des << projets de réflexion >> (voir dernière partie) pour que votre club s'oriente sur une thématique précise qu'il pourra, alors, explorer plus en profondeur.

\paragraph{Quels genre de sujets pour le club débat?}
Le club s'entend premièrement sur le but humaniste (un plus grand confort matériel et spirituel pour le plus gand nombre) et un but politique aussi: former les citoyens de demain à un regard critique et à un certain projet de réflexion. Le club du lycée Monge, par exemple, s'est accordé sur certains points qui mènent sa réflexion. Elle s'est fixée comme réalisation ultime des système sociaux l'anarchisme, qu'elle a définit comme la capacité des individus à se contraindre eux mêmes pour le bien commun. Cependant, cet idéal, comme la plupart, étant irréalisable en pratique, ils immaginent à quoi devrait ressembler le << système de transition >> qui aurrait comme dessein principal de tendre nfiniement vers l'anarchie. Et vous, quel sera votre grand projet, au delà des principes de base déjà édictés dans cette charte?

\section{Les différents systèmes de prise de parole}
\paragraph{La prise de parole spontanée}
De loins la plus commune, elle consiste à prendre la parole librement, sans couper l'interlocuteur actuel (ou alors en lui faisant remarquer qu'il ferait mieux d'abréger).
\paragraph{La prise de parole en passage de relais}
Plus organisée que la précédente, elle est toujours possiblement non-hiérarchisée mais impose à ceux qui souhaitent prendre la parole à la suite de celui qui est en train de parler de lever la main et d'attendre que celui-ci la leur donne. Conseil: notez sur un bout de papier ce que vous vouliez dire, car il n'est pas rare que les différentes prises de paroles qui occurent avant la votre ne vous perturbent et vous faisant oublier ce que vous aviez en tête. Les prises de paroles sont classées en deux catégories (qui donnent toutes deux lieu à des signes de mains différents que chaque club déterminera):
\begin{itemize}
 \item Les objections
 \item Les rebondissements
\end{itemize}
\subparagraph{Les objections}
Ce sont des interruptions courtes qui doivent êtres traitées en priorité car elles offrent des réflexions << à chaud >>. Celui qui parle doit au plus vite donner la parle à une objection:
\subparagraph{Les rebondissements}
Celà signifie que le membre a envie de réagir ou de lancer un autre sujet, en somme, il veut que la parole lui revienne. Il s'agit pour celui qui parle de terminer ce qu'il avait à dire pour laisser son camarade rembrayer.
\paragraph{Attribution de la parole par l'oeil}
De manière plus centralisé, celà peut être à l'oeil de déterminer l'ordre de parole en désignant ceux qui lèvent la main (nous distingons toujours les objections des rebondissemnts, pour que l'oeil puisse faire des choix plus pertinents). C'est une certaine responsabilité qui nécessite de l'attention pour que le débat reste équilibré.




%%%%%%%%%%%%%%%%%%%%%%%%%%%%%%%% Annexes %%%%%%%%%%%%%%%%%%%%%%%%%%%%%%%%
\part{Annexes}

\section{Créer son antenne CDEpisPo en dix étapes faciles!}
\begin{description}
 \item [Etape 1 :] Trouvez un contact dans le club débat d'épistémologie politique, qui vous servira de soutien ainsi que de taupe. Débusquez aussi quelques camarades intéressés (au moins trois).
 \item [Etape 2 :] Fournissez vous un exemplaire récent de la charte et lisez le entièrement.
 \item [Etape 4 :] Connectez vous sur le site. Furetez bien, consultez les forums publics, les vidéos, imprégnez vous de l'esprit, de l'ambiance et des visions et codes du club. Ensuite, créez une candidature en ligne, et faites une demande de création de club. Celà fait, votre demande devra être validée et acceptée et un archiviste, qui vous contactera pour vérifier que vous êtes aptes à créer ce nouveau club.
 \item [Etape 4.5 :] Participez à une séance du club débat en tant que spectateur.
 \item [Etape 5 :] Grâce aux conseil de votre tuteur et votre abnégation, la paperasse devrait être réglée rapidement et vos premières séances officieles démarrer promptement! Bon courage, faites preuves de sang froid et de rationnalité quand vous traitez des affaires du club, vous avez désormais le status de TYRAN (Téméraire Yeoman, Représentant vis-à-vis de l'Administration Nationale) dans votre club.

 \item [Etape 6 :] Vous avez créé le club, mais diriger le club d'une main de fer est contraire à ses principes, il vous faut donc rapidement déléguer votre pouvoir. Pour commençer, procédez au vote de votre règlement local, qui contiens toutes les variantes des règles que vous avez choisis de suivre, voir plus (se référer à la section "règlement local"). Si vous le désirez, affichez-le lors de vos réunions pour qu'il soit claire pour tous les membres. 
 \item [Etape 7 :] Procédez à la première élection de l'Archiviste. Choisissez votre projet de réflexion.
 \item [Etape 8 :] Pour votre première véritable séance, veillez bien à ce que tous vos membres sachent à quoi s'attendre, et qu'ils soient en mesure de respecter la charte.
 \item [Etape 9 :] ???
 \item [Etape 10 :] Profit!!!
\end{description}

\section{Conventions et nomenclature}

\subsection{Organiser les séances en groupe séparés}
Lorsque le club fonctionne au delà de l'échelle d'un simple groupe de réflexion, il est nécessaire de trouver un moyen de rester efficace tout en partitionnant le club. Dans le cadre d'un projet de réflexion, chaque cellule ne doit pas dépasser le nombre de dix personnes. Voici donc le genre d'organisation pyramidale qui est conseillée d'adopter alors:
\paragraph{Discours sur la méthode}
Il y a en effet plusieurs moyens de procéder au niveau matériel, avant même de décider des règles, classées en deux catégories: Les solutions informatiques et les solutions physiques. Ci-dessous une liste des différents moyens imaginables:
\begin{itemize}

 \item Informatique :
 \begin{itemize}
  \item Utilisation d'un logiciel spécialisé programmé par des membres
  \item Utilisation d'un logiciel tierce (organisation ou tirage au sort)
 \end{itemize}
 
 \item Physique :
 \begin{itemize}
  \item Tirage au sort planifié par concertation, avec un ordre prédéfini.
  \item Utilisation d'un objet pour désigner le membre par tirage notemment (ex: faire tourner au centre d'un cercle de membres un stylo, veiller à ce qu'il face environ un tour ou plus avant de s'arrêter).
  \item Utiliser des feuilles avec des listes de nom pour comptabiliser les membres à la place d'un ordinateur.
 \end{itemize}
\end{itemize}

Les différent moyens de réaliser le vote:
\begin{description}
 \item [Le vote à main levé] Facile et connu de tous, il a comme désavantage d'empécher l'anonymat et d'être plus délicat pour les propositions avec plus de deux issues.
 \item [Le vote plurivoque] A chaque proposition possible est attribuée un numéro. Puis ce numéro est tiré au sort.
\end{description}

\subsection{Différent types de majorités}
Voici les conventions des noms données aux différends types de majorités qui permettent une plus grande précision et commodité dans l'utilisation du vocabulaire de majorité au sein du club.
\begin{description}
 \item [Majorité par division] - 50\%
 \item [Majorité simple] + de 50\%
 \item [Majorité supérieure] + 65\%
 \item [Majorité représentative] + 80\%
 \item [Majorité absolue] + 95\%
 \item [Unanimité] 100\%
\end{description}

\subsection{Les manières officielles de nommer le club}
\begin{description}
  \item [Club débat] Tout simplement pour ne pas intimider les élèves et les officiels (personnel enseignant et éducatif par exemple).
  \item [Club débat d'épistémologie politique] Le nom complet n'est nécessaire qu'en face de ceux qui le désirent où devraient le connaître (comme les médias ou ceux qui peuvent aider à la propagation de nos idées, ou qui sont juste intéressés). Le prononçer plus souvent ne vous donnera pas l'air plus intélligent, du moins pour la plupart des gens que vous pourriez croiser. 
  \item [CDEpisPo] (prononcez kdépispo)  Court et très officiel. Essayez d'utiliser cette abrévation à chaque fois que vous voulez désigner le club précisément (par exemple par écrit) et que le nom complet ne s'impose pas.
\end{description}

\section{Patrons et conseils d'organisation}

\subsection{Propositions concernant les votes}
Les clubs doivent régulièrement procéder à des votes, que ce soit pour les conclusions, les modifications de la charte, l'élection de nouveaux membres ou de toute stratégie qui ne fait pas l'unanimité.
Premièrement, le vote doit toujours être précédé d'un débat. Ensuite, si le vote est binaire, il est demandé à ceux qui sont d'avis minoritaire de se justifier après que les raisons de la majorité aient été résumées succintement. Si le vote accepte plus de deux issues, il est demandé aux partisans de chacque avis de résumer leurs arguments et de débattre un peu. Enfin est éxécuté le vote qui peut être réalisé de diverses façons décrites ci-dessous.

\subsection{Proposition sur l'organisation d'un texte de conclusion}
\paragraph{Les conclusions}
L'Archiviste a comme travail de faire la synthèse des conclusions des différentes séances via la création d'un document succint et sobre qui fait ressortir clairement, selon les principes des clubs, les différents points abordées, les arguments et leurs limites. C'est pour celà que je vous expose ici un standard possible que je vous conseil personellement d'utiliser.
\paragraph{Organisation hiérarchique}
Pour soumettre son texte, le postulant doit veiller à hiérarchiser ses propositions en fonction de leurs liens de parenté et de les identifier chacunes par des numéro, de sorte que les membres puissent facilement et sans ambiguités déterminer quelle proposition dépend de quelle autre
\begin{itemize}
 \item 1. Chaque proposition doit être préalablement numérotée via des chiffres (arabes: 1,2,3... ou romains: I, II, III...) ou des lettres (min: a,b,c ou maj: A,B,C) suivant cette hiérarchie:
 \item 2. D'abord les chiffres romains, puis les lettres majuscules, puis minuscules, et enfin les chiffres arabes.
 \item 3. Chaque proposition doit être développé logiquement selon ses arguments, sous arguments et justifications. [Ju] Ainsi, les différents points pourront être traités méthodiquement. Les différentes proposition doivent posséder une justification. 
 \item 4. Les différentes parties du sujet, ainsi que leurs propositions sont traités de façon indépendantes. [Ju] Ainsi, l'étude du texte se fait en autant de balayages successifs que nécessaires, en commençant par les proposition les plus haut plaçées dans la hiérarchie, pour finalement terminer par les plus basses, ainsi on s'assure de traiter en premiers les piliers fondamentaux du texte avant de l'analyser plus en profondeur, ce pour une question de méthode et donc d'efficacité.
 \item 5. En complément des propositions, des textes de justifications, précédés de la balise [Ju] doivent si possible être ajoutés.
 \item 6. La balise [Rem], pour remarque est une autre balise de pré-paragraphe qui permet au relecteur de demander au club prendre en considération certains points qu'il considère comme incomplets.
 \item 7. La balise [Prp], pour proposition sert à énoncer un point d'ordre éthique plus qu'empirique sur la vision de l'état vis à vis de certains points précis.
\end{itemize}

\section{Argumentation et réthorique}
\subsection{Les pièges d'argumentation de base, les décrypter et se défendre}

\subsubsection {Les grands principe d'analyse de l'argumentation}

\begin{itemize}
 \item La charge de la preuve
 \paragraph{}
 {\scshape Certains des extraits suivants sont tirés entièrement ou en partie du site} http://www.zetetique.fr/index.php/dossiers/288-zetetique-ufologie
 \paragraph{} 
 [...] Si tout est possible, tout ne peut pas être : il faut donc démontrer positivement l’existence de ce qui est, puisque démontrer l’inexistence de ce qui n’est pas est impossible. A la base de cette idée figure une logique simple, fondée sur le principe suivant : \textbf{« l’absence de preuve n’est pas preuve de l’absence »}. Pour illustrer ce concept, on rappellera ici un exemple bien connu des zététiciens, celui des corbeaux blancs. En effet, pour prouver que les corbeaux blancs n’existent pas, je devrais parcourir la Terre entière dans ses moindres recoins, et ce simultanément pour m’assurer que l’espiègle volatile albinos ne se soit pas montré ailleurs lorsque j’avais le dos tourné. Pour faire cela, il me faudrait être omnipotent, omniprésent et omniscient, toutes caractéristiques évidemment hors de ma portée. Que je n’aie pas de preuve de l’existence des corbeaux blancs ne signifie pas qu’il n’en existe nulle part. Par conséquent, en pareil cas, il sera infiniment plus simple de démontrer positivement l’existence des corbeaux blancs, en produisant un spécimen de cette espèce par exemple.
 \subparagraph{La charge de la preuve est à celui qui avance un élément nouveau}
 De ce raisonnement découle le principe de la charge de preuve : c’est à celui qui affirme l’existence de quelque chose d’en apporter la preuve. Dans le même ordre d’idées, on doit considérer que ce qui n’est pas démontré est inexistant par défaut. Si je partais du principe inverse, en affirmant par exemple, sans preuve, que les corbeaux blancs existent jusqu’à preuve du contraire, je me trouverais rapidement dans une impasse, étant justement dans l’incapacité logique d’apporter la démonstration qu’ils n’existent pas. Or, il est aussi possible qu’il n’y ait pas de preuves de l’existence de tels oiseaux tout simplement… parce qu’ils n’existent pas. Une véritable ouverture d’esprit oblige aussi à tenir compte de cette éventualité ! Dire « les corbeaux blancs n’existent pas jusqu’à preuve du contraire » est donc la seule posture qui englobe toutes les possibilités, y compris l’inexistence pure et simple. Évidemment, ce raisonnement fonctionne aussi en remplaçant « corbeaux blancs » par « visiteurs extraterrestres ». 
 \paragraph{Réfutbilité}
 Si, comme on l’a vu, toutes les possibilités sont théoriquement envisageables, elles ne sont en revanche pas toutes égales entre elles. Certaines sont vérifiables – c’est-à-dire qu’elles se basent sur des éléments ou des faits qui nous sont accessibles, ou qui peuvent être reproduits – et d’autres non. Supposons par exemple que deux personnes viennent affirmer l’existence des corbeaux blancs, l’une sans preuves, et l’autre avec un spécimen de l’animal. La première assertion sera invérifiable, donc irréfutable et… irrecevable. Tandis que la seconde, elle, pourra faire l’objet d’une vérification : on pourra par exemple s’assurer que le volatile est bien authentiquement blanc, et non un pauvre corbeau ordinaire repeint par accident ou par malice. Ceci amène immanquablement à évoquer ce qu’on nomme la « réfutabilité » d’une hypothèse. Due au philosophe des sciences britannique d’origine autrichienne Karl Popper, ce critère est un de ceux permettant de déterminer si une hypothèse est scientifique ou non. Celles qui font appel à des éléments invérifiables ou ne pouvant être reproduits ne peuvent être considérées comme scientifiquement solides. [...]
 
 \item Rasoir d'Occam
 \paragraph{Le rasoir d'Occam}
 [...] D’autres font appel à plus ou moins de conditions, ou de suppositions préalables, dans leur formulation. Typiquement, l’hypothèse voulant qu’une intelligence extraterrestre soit à l’origine des ovnis en nécessite un grand nombre : « si » il existe une vie ailleurs que sur notre planète, « si » il s’agit d’une vie intelligente, « si » elle a produit une civilisation technologique, « si » celle-ci est parvenue à voyager dans l’espace, « si » elle est contemporaine de la nôtre, « si » elle a trouvé le moyen de vaincre les énormes distances de l’espace interstellaire, « si » elle nous a trouvé dans l’immensité de l’univers… Elle est donc « coûteuse » car elle fait intervenir de nombreuses inconnues. En ce sens, elle va à l’encontre du principe d’économie d’hypothèse – encore appelé « rasoir d’Occam » – qui veut qu’en présence de deux explications, il convient de privilégier la plus simple, celle faisant appel au moins de suppositions, en premier lieu parce qu’elle sera généralement plus facile à vérifier et la plus probable – ce qui ne signifie pas obligatoirement qu’elle est la plus vraie.
 \paragraph{Curseur de vraisemblance}
 De ce principe en découle un autre : « une affirmation extraordinaire requiert une preuve [plus qu'ordinaire] ». Cela ne signifie pas qu’un phénomène « paranormal » nécessite obligatoirement une preuve qui serait elle aussi « paranormale » (dans le cas des extraterrestres, une sonde en panne ou un spécimen observables suffiraient), mais plutôt qu’on n’attendra pas de ce genre d’hypothèse le même degré de preuve que d’une autre moins coûteuse. Prenons l’exemple de la « photo surprise » de Bar-sur-Loup (Alpes-Maritimes), prise en 2006 : dans la mesure où l’existence des pigeons est avérée et la présence de ces volatiles à proximité au moment où la photo a été prise l’est également, on ne demandera pas aux tenants de cette explication de la démontrer jusqu’à la moindre plume de l’oiseau. Au contraire, l’hypothèse faisant de « l’objet mystère » [en réalité un pigeon] de la photo un vaisseau extraterrestre devra l’être bien davantage, compte tenu des incertitudes qui l’entourent et de ses implications sur nos connaissances et notre vision de l’univers. On parle aussi de « curseur de vraisemblance  » pour désigner ce principe : plus la vraisemblance (aussi appelée « plausibilité antérieure » par certain sceptiques) d’une affirmation sera faible en regard de nos connaissances actuelles, plus elle devra être étayée pour être acceptée comme vraie.
 
 \item Parenthèse sur le doute en science
 \paragraph{}
 Pour autant, peut-on dire qu’armé de ces principes épistémologiques, on pourra aboutir à des connaissances certaines ? Non. En science, on peut toujours douter de quelque chose, à cause de la subjectivité de l’observateur, ou d’un possible défaut d’instrumentation… Objectivité et certitude absolues n’existent pas : la démarche scientifique n’accouche que de conclusions valides seulement jusqu’à preuve du contraire.

 \paragraph{} 
 Mais si cette objectivité est inaccessible, on peut en revanche s’en approcher. On doit donc s’appliquer à atténuer son contraire – c’est-à-dire la subjectivité – par une méthodologie adéquate, au même titre que dans un protocole expérimental en double aveugle, destiné à réduire la subjectivité de l’expérimentateur dans le recueil et l’interprétation des résultats d’une expérience, et dont la reproductibilité permettra de réduire les risques de biais liés, par exemple, à un défaut d’observation. [...]
\end{itemize}

\subsubsection {Les tentatives d'attaque}
\begin{itemize}
 \item L'homme de paille
 \paragraph{L'homme de paille}
 \item L'extrapolation
 \paragraph{L'extrapolation}
Quand un adversaire est acculé, il peut arriver qu'il tente de dévier du sujet initial pour vous attaquer là où vos arguments précédents ne s'appliquent plus, en changeaut de contexte. On peut comparer ce genre de malhonêteté intellectuelle à l'homme de paille, sauf que cette fois, c'est vous le complice. J'explique A et B dicutent de la guerre: A - "Je pense que le combat des Kurdes est légitime, ils forment en effet une entité culturelle séparée, etc.. etc" B - "Sauf que les" 
Autre exemple, dans un autre registre:
A- "Le fonctionnement de l'esprit humain peut être comprit dans sa totalité." B- "Sauf que le Big Bang, ainsi que les nombres univers restent des mystères pour la science et l'épistémologie. Puisque certaines des composantes de base de l'univers sont inconnues, il est impossible de comprendre le cerveau humain." A - "..."
    Ce genre de fausse connxion logique permet, sous couvert de traiter un aspect 
    B incite A à donner un contre argument à sa proposition, car A, s'il s'est laissé entrainer par le subterfuge, pense que s'il n'arrive pas à infirmer les arguments de B, alors sa première proposition sera démontée. Or c'est faux, car B s'est contenter de créer une connexion qui possède l'apparence de la logique entre deux propositions qui n'ont en fait rien à voir, lui permettant de coincer A.
\end{itemize}

\section{Organisation du site web}

\subsection{A propos de la communication entre clubs}
Le club d'épistémologie politique aime que chaque club garde une certaine indépendance dans la gestion de ses propres affaires, d'où la grande flexibilité de la présente charte. Cependant, les clubs s'inscrivent aussi dans le cadre d'un projet de réflexion plus large, à portée universelle et politique. C'est pourquoi tous les clubs doivent essayer de garder un contact actif entre eux notamment grâce au site web pour pouvoir s'organiser ensemble et conduire à des projet de réflexion synchronisés ou des rencontres en personne interclub par exemple.

\subsection{Organisation du site web}
\paragraph{Les menus et leurs pages}
L'utilisateur tombe par défaut sur la page d'acceuil du site web. Le site propose un menu permettant d'accéder, dans cet ordre, à ces menus:
\begin{description}
 \item [Acceuil] Accès à la page d'acceuil
 \item [Articles] Le portail des publications du club
 \item [Forum global] Le forum des sujets globaux et des débutants
 \item [Pétitions] Le portail des pétitions
 \item [Charte] La charte globale ainsi que le lien vers le git
 \item [Nous découvrir] Contiens les moyens de nous contacter, des liens vers les créations des membres, la F.A.Q ainsi que les documents concernant a zététique et les vidéos explicitant et vulgarisant certains principes du club.
 \item [Connection] Se connecter
 \item [Inscriptions] S'inscrire
\end{description}


La charte définit clairement le fonctionnement et l'apparence du site, de sorte à écarter l'arbitraire des artisans du site des points importants.
\subsection{Les pétitions}
Chaque membre peut créer et consulter des pétitions dans le volet << pétitions >> du menu principal du site.
\subsubsection{Les propriétés des pétition}
Une pétition est définie par:
\begin{itemize}
 \item Une phrase courte qui en donne l'idée générale.
 \item Un texte qui explique la modification en question, et à quoi elle s'applique précisément (Par exemple s'il s'agit de la charte, préciser le hash de commit)
 \item Un autre texte qui la justifie.
 \item Sa ou ses catégories, parmi:
 \begin{description}
  \item [Charte\_tete] Modifications de la tête de la charte.
  \item [Charte] Modification ayant trait aux parties externes de la charte.
  \item [Site] Proposer une modification quelconque sur le site web.
  \item [Site] Modification des algorythme su site.
 \end{description}
 Ainsi qu'une pécision précédée par << : >>, par exemple << Charte:Modification >> :
 \begin{description}
  \item [:Modification\_forme] Proposer une manière plus juste de dire les choses.
  \item [:Modification] Proposer une modification qui touche au fond, qui change le sens du sujet.
  \item [:Orthographe\&Grammaire] Correction orthographique ou grammaticale.
  \item [:Ajout] Proposition d'ajout d'un ou plusieurs articles.
  \item [:Suppression] La suppression bonne et simple de sections entières.
  \item [:Algorythme] Traitant de la modifications de parties mathématiques et algorythmiques.
 \end{description}
 \item L'historique de ses propres modifications par l'auteur, pour que les signataires soient notifiés de toute modification de la proposition.
\end{itemize}

\subsubsection{Comment les pétitions sont elles menés à terme}
Une fois qu'une pétition est créée, elle va passer par 3 phases avant d'être définitivement appliquée ou rejetée
\begin{enumerate}
 \item La pétition doit d'abord récolter un nombre suffisant de voix pour pour passer en seconde instance.
 \item La pétition doit désormais récolter un nombre suffisant de voix pour pour et suffisamment peu de voix contre, sachant que le vote blanc existe. Ce nombre est détaillé pour chaque type de pétition plus bas.
 \item Une fois validée (elle est tout simplement jetée aux archives si elle est refusée par les membres), un archiviste marqué comme disponible (voir subsection profil) choisis au hasard devra s'acquiter de la tache. S'il ne le fait pas dans un intervale de deux jours, la tâche est dévolue à un autre archiviste suivant le même procédé.
 \item L'archiviste devra soumettre sa modification à l'ensemble des autres archivistes disponibles en réunion, et il doit enfin soumettre lui même sa modification à un vote sur deux jours en demandant aux membres s'ils jugent que son exécution de la demande était conforme au sujet d'origine de la pétition, la modification devant receuillir au moins la majorité supérieure pour être acceptée.
\end{enumerate}
Par défaut, une pétition n'est éligible en seconde instance que si 20\% des membres du club la trouvent importante (la trouver importante ne signifie pas forcément qu'on vote pour elle, mais pour sa pertinence et les questions qu'elle soulève), et est automatiquement archivée au bout de deux semaines d'existence.
Par défaut, une pétition n'est éligible que si elle rassemble au moins une majorité simple de votes pour et moins de 30\% de votes contre sur au moins 20\% des membres sur trois jours. Ci dessous les exeptions par catégorie:
\begin{description}
 \item [Charte\_tete] Pour faire changer la tête de charte, il faut avoir au moins la majorité représentative de voix ainsi que moins de 16\% de voies contre.
\end{description}

\subsubsection{Comment sont affichés les pétition?}
\paragraph{Affichage et priorité d'affichage}
Les pétitions sont affichés sous forme de liste. Chaque pétition contiens les attributs suivants:
\begin{itemize}
 \item Une brève description de son sujet
 \item Le type de la pétition
 \item Un lien pour en savoir plus
 \item Un lien pour voter la pétition
 \item Le nombre de vote pour/contre
\end{itemize}
\paragraph{L'algorythme de tri des pétitions}
Pour éviter que les pétitions ne soient illisible ou noyés dans la masse, le site fourni deux moyens de classer les pétitions:
\subparagraph{Les critères de recherche}
A la disposition de l'utilisateur quand il consulte les pétitions dans le panneau pétition (panneau qui n'affiche d'ailleurs qu'un nombre défini de pétition par pages, par défaut 15 mais ce nombre peut être modifié par l'utilisateur). Premièrement, ce panneaux contiens deux boutons, un pour afficher les pétitions en première instance et un autre pour les pétitions en seconde instance. Les deux instances partagent cependant le même type de référencement:
\subparagraph{L'algorythme de valorisaion automatique}
Chaque pétition se voit attribuer un certain nombre de points auquels vont s'ajouter au fil du temps des modificateurs qui permetterons de garder l'équilibre entre d'un coté le soucis de visibilié pour les nouvelles proposition ainsi que le privilège pour les grands débats de rester visible plus longtemps. Voici la description des différends critères de modification des pétitions:
\begin{itemize}
 \item Premièrement, toute voix ajoutée à la pétition, peu importe son avis, lui fait gagner 1 point.
 \item Ensuite, les nouvelles pétitions bénéficient d'un modificateur aléatoire du nombre de points qui lui rajoute entre 20\% et 90\% de la moyenne des voix des 40 meilleures pétitions en terme de score total actuellement en lice points à chaque fois qu'on lui demande son nombre de points, ce quota passe à 10\% et 40\% au bout de 24 heures pour finalement disparaitre au bout de trois jours à compter du post du message.
\end{itemize}

\subsection{Gestion du compte et du profil des membres}
\paragraph{Cinq types de profil}
\begin{description}
 \item [Visiteur] Il peut participer aux forums, mais n'a pas le droit de posséder d'icone et ne peut naturellement pas voter, ce qui permet de le différencier du premier coup d'oeil des membres, en plus du fait que son icone ne soit pas encadré par la bordure de couleur des membres. Ils peuvent juste éditer leur pseudo une fois. Ils peuvent changer leur adresse email. Changer leur mot de passe. Changer leur signature. Il n'est pas possible de créer deux comptes avec le même mail.
 \item [Membre] Il peut voter, participer à toutes les discussions et possède un cadre de la couleur de son club autour de son icone. Un membre est un visiteur qui a été reconnu comme membre par la majorité représentative des autres membres de son club.
 \item [Archiviste] En plus de sa fonction de membre, le status d'archiviste est donc attribué par vote. Le rate de ce vote est déterminé par les paramètres originaux entrés dans la charte locale, et ils doivent êtres renseignés sur le site par le tyran avant la première élection de l'archiviste. Pour modifier ces paramètres, il est ensuite possible de changer les pétitions. Ce status permet de procéder à l'éxécution des pétitions, d'accéder aux réunions des archivistes et de s'occuper des threads du club. 
 \item [TYRAN] C'est le même TYRAN que celui des clubs, il possède juste un titre sans pouvoirs.
 \item [Webmaster] Il possède les droits d'administration du site. Ce poste n'étant pas démocratique du tout, il est voué à se transformer quand le club sera plus développé.
\end{description}

\subsection{Modération}
\paragraph{}
Si des messages ou des publications ne respectent pas la tête de la charte, ils peuvent être signalés. Si deux Archivistes affirment le signalement, le message est supprimé et l'utilisateur reçoit un warning. En revanche, si deux archivistes infirment le signalement, alors les pochains signalements seront ignorés et leurs émméteurs prévenus du verdict des Archivistes. Si les archivistes ne sont pas d'accords mais que chaque camps contiens au moins deux pour, on considère que la question n'est pas tranchée pendant deux heures, ensuite on teste si au moins 20\% des oeils penchent d'un coté plutôt que de l'autre en faisant la différence des pourcentages pour et contre pour aboutir à un verdict final. Tans que la différence reste en deça de 20\%, le verdict est suspendu et le message est considéré comme toujours affichable.
\paragraph{}
Si un utilisateur accuse un ration de plus de 20\% de messages bannis à partir de 10 messages postés, il ne peut plus rien poster pendant une semaine, en cas de récidive, ce temps passe à deux semaines puis un mois la troisième fois. Au bout de trois fois, il peut faire l'objet d'une censure par pétition si ladite pétition récolte au moins la majorité supérieure des voix et moins de 30\% de voix contre et au moins la majorité simple de participation.


\section{Ressources et bibliographie}
\begin{itemize}
 \item Richard MONVOISIN - Pour une didactique de l’esprit critique (2007)
\end{itemize}

\end{document}
