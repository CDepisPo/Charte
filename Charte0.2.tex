\documentclass[a4paper,11pt]{article}

\usepackage[utf8]{inputenc}
\usepackage[T1]{fontenc}      % un second package
\usepackage[francais]{babel}
\usepackage[a4paper, left=3cm, right=3cm, top=3cm, bottom=3cm]{geometry}
\usepackage{default}
\usepackage{graphicx}

% \makeatletter
% \renewcommand\chapter{%
%                     \thispagestyle{plain}%
%                     \global\@topnum\z@
%                     \@afterindentfalse
%                     \secdef\@chapter\@schapter}
% \makeatother



%opening
\title{\Huge{La Charte du club débat d'épistémologie politique} \LARGE{V0.2.0}}
\author{Votée à la majorité simple ou supérieure par l'ensemble des membres de tous les clubs}
%\date{01/04/2016}

\begin{document}

\maketitle

\begin{tabular}{|p{.9\textwidth}|}
 \hline
 Cette œuvre est mise à disposition sous licence Attribution - Pas d’Utilisation Commerciale - Pas de Modification 3.0 France. Pour voir une copie de cette licence, visitez http://creativecommons.org/licenses/by-nc-nd/3.0/fr/ ou écrivez à Creative Commons, PO Box 1866, Mountain View, CA 94042, USA.
 \begin{center}
 \includegraphics[scale=1]{cc.jpg}
 \end{center}\\
 \hline
\end{tabular}
\newpage
\tableofcontents
\newpage

%%%%%%%%%%%%%%%%%%%%%%%%%%%%%%%% Les principes directeurs des clubs %%%%%%%%%%%%%%%%%%%%%%%%%%%%%%%%
\part{Les principes directeurs des clubs}
\paragraph{Introduction}
Ce club est ouvert à tous les citoyens et futurs citoyens qui cherchent à nourrir leur réflexion de l'opinion des autres et partager ses points de vue pour remettre en question ses préjugés. Ceux qui cherchent, au delà des frontières et des apparences, un monde où les hommes pourraient cohabiter d'une manière plus harmonieuse. A tous ceux ci, ce club ne leur est pas seulement ouvert, il a été fait pour eux, par des gens qui, comme tant d'autres, partagent cet idéal simple. Si vous êtes à la recherche d'autres personnes qui aiment réfléchir à comment faire bouger les choses, si vous aimez prendre le temps d'explorer le monde des idées au delà des préjugés et des apparences, nous sommes à la recherche de personnes telles que vous. En espérant que cette charte vous inspirera, bonne lecture. - Les membres du club débat d'épistémologie politique

\section{Principes Ethiques}
\paragraph{} 
Pour se préserver des préjugés et des raisonnements binaires qui enrayent tant de débats et pour donner matière à réfléchir sur le monde et sur les autres, voici ci-dessous les règles de base du club, qui fondent son identité. Parce que le dialogue est la condition indispensable à la paix sociale et à la démocratie véritable, c'est sur ce principe que se fonde notre démarche. La suite de la tête de la charte couche notre éthique ainsi que nos règles, pour que notre méthode se propage dans des cadres toujours plus larges. Les grands principes du club ont été créés sous l'influence (entre autre) de la zététique et certains courants anarchistes.

\section{Règles de base du débat}

\paragraph{Les règles de base}
Voici quelques règles que vous devez intégrer lorsque vous entrez dans l'enceinte du club. Il est important de les appliquer sincèrement pour non seulement ne pas devenir un obstacle au débat rationnel et paisible, mais pour offrir aux autres le meilleur de vous même:
\begin{description}
 \item[Règle 1 :] En entrant dans le débat, vous êtes prêts à considérer sans violence tous les arguments.
 \item[Règle 2 :] Vous cherchez à connaître clairement vos propres opinions, leurs arguments et leurs limites, pour rester humble et être prêt à les voir contredites et modifiés.
 \item[Règle 3 :] Vous vous efforcez de tendre vers l'objectivité dans l'analyse de vos propres arguments et de ceux des autres.
 \item[Règle 4 :] De part sa vocation et son éthique, le club de réflexion citoyenne ne peut tolérer la violence sous toute ses formes, vous devez la signaler et à garder un certain détachement vis à vis de vos émotions pour rester maître de ce que vous dites.
\end{description}
Pour vous guider dans l'application de ces règles, elles sont appuyées par les documents en annexe, dont la lecture est conseillée, car ils fournissent un point de départ et des ouvertures à qui veut se former à l'art de l'argumentation et de la contre-argumentation. 

\paragraph{Axiomes}
Voici nos "axiomes". Ce sont les faisceaux directeurs de notre éthique et de nos réflexions, et s'ils nous semblent relever du bon sens, il ne faut pas hésiter à les questionner et à les remettre en question (comme nimporte quelle autre partie de la charte). 

\paragraph{Les axiomes du club:}
\begin{itemize}
 \item  Le but du club est principalement pédagogique, il cherche à apprendre aux élèves à se défendre contre les argumentations qui nuisent à la pertinence du débat.
 \item  Pour permettre le consensus, nous définissons les objectifs à long termes suivants pour nos délibérations: une meilleur comprehension de qui nous sommes (pour ne pas stagner), la survie des formes de vie évoluées (a peu près toutes donc, pour préserver la vie au cas ou ce serait utile) et la satisfaction des aspirations de la majorité des hommes (pour ne pas nous nuire).
 \item  Nous essayons de tendre grâce au débat vers une vision plus objective et universelle du monde (qui parle à tous les hommes, sans distinction de culture ou de croyance).
 \item  Le club est laïque. Bien que respectant les croyances de chacun, nos idées sont, autant que possible, portées par la logique et la méthode scientifique (à ne pas confondre avec le scientisme).
 \item  Plutôt que la Vérité, nous cherchons l'option la plus raisonnable compte tenu des éléments en notre possession, tout en gardant sous la main les autres options possibles si jamais des faits nouveaux nous dirigeaient dans leur sens.
\end{itemize}

\section{A quoi sert cette charte?}

\paragraph{Qui controle la charte?}
La charte est faite pour être controlée d'une manière démocratique qui fonde sa légitimité. Pour plus d'information, renseignez vous dans la section décrivant le fonctionnement du site web dans les annexes.
\subparagraph{Légitimité de la charte}
Vous vous demandez sûrement, en tant que membre ou futur membre du club, sur quoi se fonde le légitimité de cette charte. Sachez d'abord que si la partie de la charte que vous êtes en train de lire, communément nommée tête de charte est faite pour rester relativement inchangée dans son fond, le reste de la charte peut être modifiée, et chaque club peut choisir ses propres variantes de certaines règles sans aller au delà de la charte. Chaque club ayant délibérément choisis d'être une "Antenne du club débat d'épistémologie politique", il a accepté de se soumettre aux principes de bases de la première partie de cette charte. La tête de la charte est en quelque sorte l'équivalent d'une ``constitution'', plus stable que lois, qui vont en derrière. Ceci dit, nous sommes très attachés à la démocratie, alors si vous n'êtes pas d'accord avec certains points de cette charte, vous pouvez les rendre plus légitimes en les remettant en cause sur le site du club grâce à une procédure démocratique détaillée dans la partie décrivant le fonctionnement du site web.

\paragraph{Lire et interpréter la charte}
Pour un plus grand confort de lecture, la charte est écrite en plusieurs parties indépendantes, qui vous permettent de vous y retrouver facilement. Après les principes éthiques, tout ce qui se trouve dans la charte ne doit pas être respecté à la lettre. En fait, à la création de votre club, il vous est demandé de faire des choix entre les différentes possibilités de fonctionnement pour votre club (voir la section qui parle de la création d'un nouveau club). Pour ce qui est de l'interprétation, si vous trouvez des conflits entre le contenu de la charte et sa "tête", ou d'autres parties de la charte, merci d'en faire part sur le site. Et pour régler la majorité des incohérences ou marge d'interprétation, référez vous à la tête de la charte, et agissez conformément à ses principes de base pour minimiser au possible les désaccords. En cas de désaccord, c'est le modérateur de la séance (ou << Oeil >>, voir la partie dédiée) qui décide de la marche à suivre, mais n'oubliez pas de faire part sur le site de tout problème rencontré avec la charte durant vos propres débats.

\paragraph{Origine du nom}
Le nom temporaire de "club de réflexion citoyenne" était utilisé au début. Il nous apparut au bout d'un certain temps, alors que nos buts et nos ambitions se précisaient que ce nom n'était pas suffisamment représentatif de notre projet. Nous cherchions un nom qui incarnerait notre souhait d'une réflexion plus globale sur notre monde, portée sur la politique et l'aspect analytique, méthodologique et rationnel que nous comptions porter sur nos opinions. Après plusieurs débats, nous adoptâmes finalement un nom contenant l'épistémologie, c'est à dire l'étude critique des sciences, appliquée à la politique, avec au premier plan la notion de débat. A ce jour, ce nom est l'interprétation la plus consensuelle des principes de notre club.

\section{Manque de respect à la charte}
\paragraph{Manque de respect à la charte}
L'obéissance à la charte permet de lier tous les clubs et de mettre au clair les principes de base du débat démocratique. Si un membre agit à l'encontre de la charte, et qu'il refuse de changer son comportement malgrès qu'on lui ait fait remarquer son erreur, voici quelques propositions pour pouvoir poursuivre le débat en le mettant à l'écart, proportionellement à sa faute:
\begin{itemize}
  \item 1: la faute est assez faible et que l'auteur l'assume plus ou moins, par exemple lors d'un emportement ou qu'il n'a de cesse d'interrompre ses camarades, s'il a du mal à se calmer seul tout en continuant le débat, il doit se mettre à l'écart pour se remettre en question, se calmer, et reprendre le débat dans de meilleures conditions.
  \item 2: le membre commet des fautes répétées et qu'il refuse de se remettre en question et de respecter la charte, il devient un poids pour le club. Il faut alors en discuter avec lui. Si malgrès cela il ne veux rien entendre, vous pouvez l'empêcher de parler pendant 30 minutes, ou bien l'exclure jusqu'à qu'il se décide à respecter les autres membres, selon sa personnalité et ses dispositions à re remettre dans de bonnes conditions que l'oeil devra juger.
  \item 3: cas de faute très grave (insultes à répétition, menaces, etc...), le membre doit être exclu de la séance par l'oeil, voir même renvoyé du club à la majorité simple.
  \item Pour éviter que ce ne soit pire, je vous conseille de faire attention à qui vous parrainez, sinon (les débats peuvent être éprouvants) le reste risque d'être du ressort du personnel de votre établissement, de la police, voir même de l'armée.
\end{itemize}


\newpage
%%%%%%%%%%%%%%%%%%%%%%%%%%%%%%%% Rôles et organisation humaine %%%%%%%%%%%%%%%%%%%%%%%%%%%%%%%%
\part{Rôles et organisation humaine}
\section{L'Oeil et le Scribe}
Les modes d'élection de l'Oeil et du Scribe est décidé dans la charte locale.
\paragraph{L'Oeil}
L'Oeil est le modérateur de la séance. Il a le rôle de maintenir les membres dans le sujet, de maintenir une réflexion construite et d'intervenir pour que la séance soit productive. Selon le type de séance et la charte locale, il pourra par exemple désigner qui a la parole, l'orientation du sujet en cours de séance, interrompre les hors sujets abusifs ou mettre en pause le débat quand il ne le juge plus pertinent (il doit bien évidemment justifier son interruption auprès des membres). Certains débats, comme les "débats confidentiels", peuvent se passer d'oeil. Pour l'aider dans sa tâche, le site web propose des documents vidéos et des conseils, et l'annexe, que l'aspirant Oeil doit avoir lu contiens le minimum vital pour pouvoir discerner les remarques pertinentes de celles qui le sont moins. Une bonne connaissance de la réthorique, une autorité non négligeable et une conscience rigoureuse est primordiale pour faire un bon oeil. 

\paragraph{Le Scribe}
Le scribe est le secrétaire de la séance. Son rôle n'est pas facile car il doit suivre le débat tout en le transcrivant, éventuellement en y participant dans les limites de ses capacités. Il a la tâche de retranscrire les arguments, leurs points forts et la manière dont les membres les ont reliés entre eux pour offrir à l'archiviste une fiche à partir de laquelle il pourra écrire les conclusions de séance. Ses écrits servirons également aux débats futurs qui se nourrirons des anciens au lieu de se répéter. Le moment venu, on ne regrette pas d'avoir gardé une trace de ce qui a été dit. Les membres peuvent faire part au scribe de suggestions sur ce qui devrait être noté. Le scribe doit prendre note de toutes les réflexions produites pendant la séance, ce qui implique une vitesse de frappe correcte au clavier. Il peut pour celà s'aider de toutes sortes d'outils, ordinateur, appareils photos, tablettes graphiques, feuilles, etc... Tant que l'Archiviste pourra accéder aux documents produits sans difficultés.

\paragraph{Différent types de scrutins}
Il existe de nombreuses façons de désigner qui fera oeil ou scribe, toutes sont valables et ont leurs propres arguments, laissez moi vous en proposer quelques unes:
\begin{description}
 \item[L'ordre prédéfinit :] Les noms des membres sont notés sur une feuille, puis ils sont tirés successivement en fonction de leur présence/absence.
 \item[Variante et règles supplémentaires :] Les membres peuvent être tirés au hasard puis choisis dans l'ordre ainsi généré, ou bien inversés à chaque fois. En cas d'absence d'un membre, son nom est noté comme prioritaire et il sera automaiquement élu la prochaine absence. Sauf s'il est absent régulièrement, auquel cas il faut le mettre dans une colonne séparée pour qu'il ne soit élu que durant ces séances pour ne pas être élu quasiment deux fois plus que les autres. Je vous laisse compléxifier les règles à votre fantaisie.
 \item[Le tirage au sort :] Le tirage au sort nous a tout de suite semblé le moyen le plus raisonnable de choisir les membres, car il les met statistiquement tous à égalité. Il est précisé dans la section précédente quelques moyens de procéder pour le tirage au sort.
 \item[Variante et règles supplémentaires :] Mais les statistiques sont perturbées lorsque un membre s'absente. S'il est absent souvent, il paricipera moins aux tâches de directon du club. Un moyen de pallier à ce problème est d'attribuer à chaque membre à responsabilité un nombre de point. celui-ci croit de 1 à chaque séance où il est présent, de 2 lorsqu'il est absent et est remis à 0 lorqu'il est sélectionné. Chaque point compte alors comme une chance d'être élu, et celui qui en possède le plus est par conséquent plus prompt à être élu.
 \item[L'élection :] Les membre à responsabilité peuvent aussi être élus, soit par tous les membres soit entre membres à responsibilité. Nous n'avons pas choisis ce système au lycée Monge car il créait de la concurrence là où nous jugions que chacun devait participer pour le confort de débat de tous. Mais essayez, vous nous ferez part de vos retours!
\end{description}

\paragraph{Qui peut être élu oeil ou scribe?}
Testez la méthode qui conviendra le mieux! Quelques pistes pour vos membres à responsabilités à tester:
\begin{itemize}
 \item Tout membre est par défaut un membrte à responsabilité [Ju] Il faut accepter tous les travaux, par solidarité
 \item Les membres peuvent choisir un ou plusieurs rôles (demande certainement des rêges supplémentaires pour rester juste)
\end{itemize}

\section{A propos du recrutement de nouveaux membres}
\paragraph{Le parrainage}
Il a l'avantage de permettre de garder un certain contrôle sur les nouveaux venus et de s'assurer qu'ils bénéficient à leur entrée du soutiens et des conseils d'au moins un membre. Ce système suppose qu'une personne qui veut faire partie du club trouve un des membres du club qui accepte de devenir son Parrain, et lui même son Aspirant. L'Aspirant doit ensuite participer à deux séances pour pouvoir être éligible au status de membre. A la troisième séance, un vote est effectué. L'Aspirant et son Parrain exposent les motivations de l'Aspirant et les raisons de le garder, puis il faut qu'au mois 80\% des membres soient d'accord de sa venue, auquel cas l'Aspirant deviendra membre à part entière.
\subparagraph{Déroulement du vote}
Classiquement, le vote commence à main levée en demandant qui s'oppose à la venue du nouveau membre. Les membres s'opposant à l'élection de l'ombre sont tenus de se justifier, et il peut s'en suivre un bref débat (moins de 4 min) à la suite duquel est fait un vote, procédant de la manière choisie dans la charte locale. Si l'assemblée des membres refuse l'Ombre, celui ci restera ombre jusqu'à être élu à une séance ultérieure, s'être lui même désisté, ou que son parrainage ait cessé.

\section{L'Archiviste}
\paragraph{Rôle de l'Archiviste}
L'Archiviste fait office de gestionnaire du club. C'est lui qui fait en sorte que tout soit en ordre, qui compose les conclusions de séance pour les soumettre à l'approbation des autres membres. Il est aussi responsable de la publication des articles de son club sur le site web qu'il doit relire et approuver, tiens à jour le répertoire de travail du club ainsi que les documents relatifs à son organisation. L'Archiviste est le pilier du club, celui qui travaille en backstage pour que le club fonctionne sur le long terme.
\subparagraph{A propos des conclusions de séance}
L'Archiviste est chargé de composer, entre deux séance, un document suivant une organisation stricte et hiérarchique comme décrite dans la partie portant sur l'organisation d'un texte de conclusion en annexe. Il est basé sur les notes du scribe. Le but d'une conclusion est de fournir aux membres un aperçu synthétique du fond de ce qui a été débattu pour les mettre faces à l'avancement actuel du débat, sur les questions qui ont pu être oublié en cours de débat pas été assez développés pour permettre de décider sur quel points se concentrera le prochain débat.
\paragraph{Désignation de l'Archiviste}
L'archiviste est désigné toutes les trois semaines par élection de tous les membres, mais vous pouvez piquer d'autres idées de désignations dans la partie sur la désignation de l'oeil et du scribe. 

\section{Organisations de la parole}
\paragraph{Différentes organisations}
Il est conseillé de changer l'organisation du débat en fonction du nombre de membres présents. Pour celà, voici une proposition de convention relativement arbitraire de nommage du nombre de membres en 5 catégories:
\begin{description}
 \item[2-4 membres :] Débat confidentiel
 \item[5-10 membres :] Groupe de réflexion
 \item[11-25 membres :] Petite assemblée
 \item[26-60 membres :] Assemblée
 \item[61+ membres :] Grande assemblée
\end{description}

Il est normal d'éviter de débattre directement une fois atteint un certain nombre de membres, car il est admis que le débat est humainement impossible avec autant de personnes à la fois (l'expérience montre que les membres tendent à se rassembler de toute façon en unités de réflexion plus petites au sein de cette assemblée). La compartimentation du club peut se faire à partir du statut de petite assemblée, au dessus, elle est très conseillée. Ci-dessous sont développés des conseils sur les organisations et conventions à adopter en fonction du nombre de membres que vous acceuillez.

\paragraph{Note:}
Dans tous les cas, si une séance contient la majorité supérieure des membres présents, l'Archiviste devra soumettre ses conclusions en début de séance. S'il est impossible de réunir tous les membres (ce qui est une situation compliquée, mais malheureusement possible) faites voter tous les membres (jusqu'à atteinte de la majorité supérieure) sur plusieurs séances, et comptabilisez simplement le total. Je précise cependant que le club du lycée Monge n'ayant pas eu à faire face à ce genre de problème, c'est à vous qu'il revient d'ajouter de nouvelles règles à cette manoeuvre pour la rendre plus juste, d'aucun diraient plus logique.

\paragraph{Débat confidentiel}
Lorsque vous êtes moins de cinq, les règles du débat importent peu: en effet, c'est rarement lors de ce genre de séances que l'on imposera des changements importants pour le club. En revanche, vous pouvez utiliser ces séances pour parler de sujets plus décontractés qui sont parfois hors du projet de réflexion actuel, pour se changer les idées. Ou alors, débattre sur des points de la charte. Plus mûres, vous pourez proposer vos idées plus construite aux autres membres du club pendant une séance plus fréquentée. Il est conseillé d'élire un Scribe (voir la section afférente), la présence d'un oeil n'est pas indispensable. Le système de prise de parole spontanée est conseillé.

\paragraph{Groupe de réflexion}
C'est là que le club devient réellement opérationnel. Pour l'organisations, jetez un coup d'oeil à la section nommée "Déroulement d'une séance de débat". Il est conseillé d'utiliser une organisation de type oeil et scribe pour ce genre de débat. 

\paragraph{Petite assemblée}
Bien qu'il soit toujours possible de tenir un débat classique, le grand nombre d'intervenants potentiel ralentira le débat et en laissera beaucoup de coté. Il est donc conseillé dans ce cas de fractionner le club en deux, puis de mettre en commun les conclusions pour les débattre à nouveaux. Cette manière de faire permet de traiter de deux sujets à la fois, ou bien de faire une séance recherche en même temps qu'une séance débat. N'oubliez pas cependant que les conclusions doivent être votés à la majorité représentative et que la majorité supérieure des membres doit être présente. Il est cependant, notamment durant les séances spéciales invités possible de garder le club uni, car le débat est recentré autour d'une seule personne.

\paragraph{Assemblée}
Si vous possédez autant de membres dans votre club, félicitations. S'ils peuvent venir, tant mieux. Je conseille en cas de séance débat et/ou recherche de partitionner le club en 3 à 7 groupes de réflexion.
Seuls les votes de conclusions ainsi que les élections de nouveaux membres se feront avec la totalité des membres.

\paragraph{Grande Assemblée}
La grande Assemblée suit par défaut les mêmes règles que l'Assemblée, le club de réflexion citoyenne n'ayant encore jamais eu à traiter un tel nombre de membres présents simultanéménts.

\section{Mettre en commun les résultats d'une réflexion}

\paragraph{Quand le club est partitionné}
Dans ces cas, les différetes fraction du club peuvent soit chacunes travailler sur des thématiques différentes (ou plusieurs aspects de la même thématique), soit chacunes débattre du même sujet. Dans ce cas, il est probables que chacunes des branches arrive à être en désaccord sur certains points. Pour "Trancher", il faut planifier durant une prochaine séance de refaire un débat, mais avec des représentants de chacuns des groupes de réflexions, généralement l'oeil et le scribe, sous la direction de l'Archiviste qui fait office d'oeil. Ce débat se déroule comme un débat normal, sauf que suite à la mise en commun, les conclusions de chacunes des branches de réflexion sera conservée avec les conclusions du débat final, pour clairement comprendre comment les opinions des un et des autres se sont modifiées. Si vous êtes vraiment très nombreux, répétez ces étapes sur plusieurs strates.

\section{Hiérarchie des membres}
Voici la liste de différents grades possibles pour les membres:
\begin{description}
 \item[Membres à responsabilité] C'est un membre qui accepte d'être éligible aux fonction de scribe et d'oeil.
 \item[Membre] Le membre a droit de prticiper aux votes, mais n'est pas éligibles aux tâches d'oeil ou de scribe.
 \item[Ombre] L'aspirant est parrainé par un membre et apsire à en devenir moyennant deux séances d'observation. Durant celles-ci, il peut participer au débat, mais pas aux votes.
 \item[Invité] Une personne extérieur a club qui est invité temporairement de part son status d'expert ou plus généralement son utilité dans le cadre d'un débat précis. L'invité est là pour débattre et doit se plier aux règle de débat établis, mais ne participe pas aux votes.
 \item[Spectateur] Peut observer, poser des questions quand autorisé, mais sans plus. Nimporte qui, pour peut qu'il ne dérange pas trop le débat peut être spectateur.
\end{description}


\newpage
%%%%%%%%%%%%%%%%%%%%%%%%%%%%%%%% Organisation des Séances %%%%%%%%%%%%%%%%%%%%%%%%%%%%%%%%
\part{Organisation des séances}

\section{Chartes locales}
Les clubs doivent décider lors de leur première séance à de la création d'une charte locale, contenant les modalités des règles adoptés par cette antenne du club. La charte se compose de la première tête de la charte officielle, inchangée, suivie des modalités adoptés. Une modalité d'écrit comme suit:
[Nom du point soumis à modalité] : [Nom de la modalité choisie]; [Précisions concernant les paramètres de la modalité, s'enchainent par des virgules]
exemple: Elections de l'Oeil : Ordre prédéfini; dans l'ordre alphabétique des noms de famille

\paragraph{}
Ci dessous la liste des points de cette charte soumis aux modalités:
\begin{description}
 \item [Mode de recrutement des nouveaux membres]
 \item [Mode d'élection de l'Oeil]
 \item [Mode d'élection du Scribe]
 \item [Mode d'élection de l'Archiviste]
 \item [Durée d'élection de l'Archiviste]
 \item [Organisation des membres au sein du club]
\end{description}

\section{Déroulement d'une séance de débat}
Ci dessous le shéma type d'une séance de débat normale:
\begin{enumerate}
 \item Phase d'installation.
 \item Temps où tous les membres peuvent proposer des idées sur nimporte quel sujet.
 \item On procède à l'élection de l'Oeil et du Scribe. L'Oeil est responsable du bon déroulement de la suite de la séance.
 \item S'il y a lieu à une conclusion, l'Archiviste présente son document et fais relire, le soumettant à l'approbation et la critique.
 \item L'oeil récapitule, éventuellement à l'aide d'autres membres l'avançée du projet de réflexion, puis annonce le sujet qui avait été décidé et donne la parole aux premiers intervenants qui vont ouvrir le débat.
 Normalement, cette organisation ne devrait pas dépasser les 30 minutes, et dure moins dans les clubs plus expérimentés.
 \item Le débat s'écoule, et s'il dure plusieures heures, l'oeil et/ou le scribe peuvent décider de procéder à un nouveau tirage au sort pour pouvoir débattre en tant que membres ordinaires, et effectuer un roulement à raison d'un roulement par heure
 \item A la fin de la séance, c'est à dire au moins dix minutes avant la fin, le scribe récapitule l'avancement du projet de réflexion durant cette séance. 
 \item Est ensuite débatut puis déterminé le ou les sujet du débat de la semaine prochaine, conformément à l'avançée courante de la thématique du projet de réflexion.
 \item Enfin, chacun peut proposer ses avis sur cette séance et ses idées pour l'améliorer.
 \item Fin de séance, on range la salle, la séance est close.
\end{enumerate}

\section{Déroulement des différent types de séances pour traiter de différent sujets}
Je vous propose ici une liste non exhaustive de types de séance que vous pouvez planifier:
\begin{description}
 \item [Les séances de débat :] C'est de ces séances que traite principalement la présente charte, et c'est à ce moment que toutes les réflexions séparées sur un sujet seront rassemblées pour formuler des conclusions et avançer dans la réalisation d'un univers social plus juste.
 \item [Les séances de recherche :] Durant ces séances spéciales, les membres se retrouverons dans une salle informatique si possible dédiée pour pouvoir faire des recherches ensemble. Ils s'organiserons pour trouver des arguments, des réponses sourcées et précises à certaines de leurs interrogations ainsi que des pistes de réflexions en vue de préparer une séance future. Elles se déroulent comme des séances classiques, mais sans scribe (chacun étant considéré comme son propre scribe). Le choix d'un oeil n'est pas obligatoire mais peut dans certains cas aider à s'en tenir à l'objectif originel de la séance. La communication et l'organisation sont des facteurs de réussite décisifs d'une bonne séance de recherche.
 \item [Les séances conférences du club :] Parfois, le club peut avoir à se présenter ou à animer une conférence portant sur certaines de ses conclusions. Celà dépend du type de séance qui sera montré. Typiquement, il peut s'agir d'une présentation suivi d'une séance classique publique. Dans ce cas, l'organisation peut s'apparenter à une séance classique, mais au final vous devrez juger au cas par cas selon l'animation que vous planifiez. Il doit alors s'organiser pour trouver des moyens de communication (diapositives, démonstration, exercices logiques) qui inciteront l'auditoire à la curiosité pour permettre de partager l'esprit du club et créer l'espace de quelques heures un ilot de réflexion citoyenne dans un monde de fous. 
 \item [Les séances spéciales invités :] Si le club invite parfois des personnes externes pour entrainer la discussion plus loin, grâce par exemple à la présence de personnes directement concernées par le sujet, il arrive que la personnalité soit si importante ou primordiale au débat qu'une variante de la séance de débat habituelle est ouverte. Elle met au centre de la discussion l'Invité, qui n'est plus alors soumis aux règles concernant les limites du temps de parole ou les interruptions. La présence d'un oeil dans ce type de séance n'est pas obligatoire. En revanche, on imagine mal une séance aussi importante dépourvue de scribe. La configuration de la salle peut être modifiée pour l'invité (estrade, chaises, hémicycle pour mieux le voir et l'entendre...).
\end{description}

\section{Choix des sujets à traiter}
\paragraph{Intro}
C'est un choix primordial, puisque le débat constitue le coeur du club. Je liste ici les différentes catégories de sujets qui sont de nature à être traités, conformément à la définition du nom du club, l'épistémologie. De fait, lorsque vous comptez débattre de sujets, il est bon de les classer en catégories, et d'organiser des << projets de réflexion >> (voir dernière partie) pour que votre club s'oriente sur une thématique précise qu'il pourra, alors, explorer plus en profondeur.
\paragraph{Quels genre de sujets pour le club débat?}


\section{Mener à bien un projet de réflexion}
Pour s'exercer, le club structure sa pensée dans un ou plusieurs projets de réflexion annuel.

\paragraph{}
Un projet de réflexion représente un sujet général assez vaste qui servira de fil directeur au fil de l'année. Le but du projet est d'être assez fédérateur pour motiver la majorité des membres et général pour pouvoir faire intervenir tous les domaines de l'épistémologie.
\subparagraph{}
Ensuite, une fois le projet de réflexon débatu et voté, deux débuts de séances sont dédiés au choix d'un certain nombre de sujets qui découlent du sujet principal. Une estimation du temps de traitement en heure de chaque sujet doit être adjointe à leur consignation, et il est important d'en proposer un grand nombre.
\paragraph{}
Au début d'un nouveau projet de réflexion, le début d'une séance (après l'élection de l'oeil et du scribe) est pris pour débattre du nouveau projet à choisir. Premièrement, les membres vont se concerter entre eux et noter sur papier leurs premières idées, puis supprimer celles qu'ils jugent impertientes en regard de la problématique précédement étudiée. Ensuite, chaque membre sera invité à donner ses problématiques sur le tableau, puis on procèdera à un vote en deux tours.
Au premier tour, chaque participant possède la racine carrée du nombre total de proposion arrondie au nombre le plus proche, et vote pour des propositions différentes. S'il y a égalité, un second tour est réalisé, mais cette fois avec un seul vote par personne, ou alors de la même manière que le premier tour. S'il y a encore égalité, on tire au hasard une des possibilités. 
Enfin, il y a un débat concernant les différents sous sujets de la problématique, puis on décide l'ordre des prochaines séances et leur type, avant de commencer la première séance de ce projet.
\subparagraph{}
A la fin d'un sujet, une séance spéciale de conclusion remet en perspective le travail fournit et la valeur des conclusions, définit les modifications apportées à la charte et le degré de réussite des débats. C'est l'occasion de revenir sur les points qui ont bien marchés et sur ceux qui ont empéché le bon débat. Une fois la discution sur ce projet close, on peut soit faire une séance un peut plus abstraite, de philosophe par exemple sur un point qui a laissé les membres pensifs, ou bien commencer un nouveau cycle.

\section{Différents systèmes de prise de parole}
\paragraph{La prise de parole spontanée}
De loin la plus commune, elle consiste à prendre la parole librement, sans couper l'interlocuteur actuel (ou alors en lui faisant remarquer qu'il ferait mieux d'abréger).
\paragraph{La prise de parole en passage de relais}
Plus organisée que la précédente, elle est toujours possiblement non-hiérarchisée mais impose à ceux qui souhaitent prendre la parole à la suite de celui qui est en train de parler de lever la main et d'attendre que celui-ci la leur donne. Conseil: notez sur un bout de papier ce que vous vouliez dire, car il n'est pas rare que les différentes prises de paroles qui occurent avant la votre ne vous perturbent et vous faisant oublier ce que vous aviez en tête. Les prises de paroles sont classées en deux catégories (qui donnent toutes deux lieu à des signes de mains différents que chaque club déterminera):
\begin{itemize}
 \item Les objections
 \item Les rebondissements
\end{itemize}
\subparagraph{Les objections}
Ce sont des interruptions courtes qui doivent êtres traitées en priorité car elles offrent des réflexions << à chaud >>. Celui qui parle doit au plus vite donner la parle à une objection:
\subparagraph{Les rebondissements}
Celà signifie que le membre a envie de réagir ou de lancer un autre sujet, en somme, il veut que la parole lui revienne. Il s'agit pour celui qui parle de terminer ce qu'il avait à dire pour laisser son camarade rembrayer.
\paragraph{Attribution de la parole par l'oeil}
De manière plus centralisé, celà peut être à l'oeil de déterminer l'ordre de parole en désignant ceux qui lèvent la main (nous distingons toujours les objections des rebondissemnts, pour que l'oeil puisse faire des choix plus pertinents). C'est une certaine responsabilité qui nécessite de l'attention pour que le débat reste équilibré.

\end{document}
