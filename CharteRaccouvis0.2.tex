\documentclass[a4paper,11pt]{article}

\usepackage[utf8]{inputenc}
\usepackage[T1]{fontenc}      % un second package
\usepackage[francais]{babel}
\usepackage[a4paper, left=3cm, right=3cm, top=3cm, bottom=3cm]{geometry}
\usepackage{default}
\usepackage{graphicx}
%opening
\title{\Huge{La Charte du club débat d'épistémologie politique - Version succinte} \LARGE{V0.2.0}}
\author{Votée à la majorité simple ou supérieure par l'ensemble des membres de tous les clubs}
%\date{01/04/2016}

\begin{document}

\begin{tabular}{|p{.9\textwidth}|}
 \hline
 Cette œuvre est mise à disposition sous licence Attribution - Pas d’Utilisation Commerciale - Pas de Modification 3.0 France. Pour voir une copie de cette licence, visitez http://creativecommons.org/licenses/by-nc-nd/3.0/fr/ ou écrivez à Creative Commons, PO Box 1866, Mountain View, CA 94042, USA.
 \begin{center}
 \includegraphics[scale=1]{cc.jpg}
 \end{center}\\
 \hline
\end{tabular}

Reflexion | Discussion | Ouverture d'esprit

\paragraph{Introduction}
Le club débat existe pour offrir des espaces d'échange et de discussion qui doivent permettre à tous de communiquer sans se retrouver dans une logique d'affrontement, d'agressivité. Nous sommes à la recherche de toujours plus de personnes qui aiment réfléchir à comment faire bouger les choses, et qui pensent qu'il est possible de parler de (presque) tout en étant écouté et compris.  Merci de ton intérêt pour notre charte! - Les membres du club débat d'épistémologie politique
\subparagraph{A propos de la charte complète}
Cette feuille est une version raccourcie de la charte, n'hésitez pas à jeter un coup d'oeil à la version complète!

\paragraph{Qui controle la charte?}
La charte est faite pour être controlée d'une manière démocratique qui fonde sa légitimité. Pour plus d'information, renseignez vous dans la section décrivant le fonctionnement du site web dans les annexes.
  
\section{Principes Ethiques}
\paragraph{} 
Voici ci-dessous les règles de base du club, qui fondent son identité:

\paragraph{Axiomes}
Les "axiomes" sont les faisceaux directeurs de notre éthique et de nos réflexions, et s'ils nous semblent relever du bon sens, il ne faut pas hésiter à les questionner et à les remettre en question.

\paragraph{Les axiomes du club:}
\begin{itemize}
 \item  Le but du club est principalement pédagogique, il cherche à apprendre à ses membres à se défendre contre les argumentations qui nuisent à la pertinence du débat.
 \item  Pour permettre le consensus, nous définissons les objectifs à long termes suivants pour nos délibérations: une meilleur comprehension de qui nous sommes (pour ne pas stagner), la survie des formes de vie évoluées (a peu près toutes donc, pour préserver la vie au cas ou ce serait utile) et la satisfaction des aspirations de la majorité des hommes (pour ne pas nous nuire).
 \item  Nous essayons de tendre grâce au débat vers une vision plus objective et universelle du monde (qui parle à tous les hommes, sans distinction de culture ou de croyance).
 \item  Le club est laïque. Bien que respectant les croyances de chacun, nos idées sont, autant que possible, portées par la logique et la méthode scientifique (à ne pas confondre avec le scientisme).
 \item  Plutôt que la Vérité, nous cherchons l'option la plus raisonnable compte tenu des éléments en notre possession, tout en gardant sous la main les autres options possibles si jamais des faits nouveaux nous dirigeaient dans leur sens.
\end{itemize}

\section{Règles de bases du débat}

\paragraph{Les règles de base}
Voici quelques règles que vous devez intégrer lorsque vous entrez dans l'enceinte du club. Il est important de les appliquer sincèrement pour non seulement ne pas devenir un obstacle au débat rationnel et paisible, mais pour se mettre en disposition de pouvoir offrir aux autres le meilleur de vous même:
\begin{description}
 \item[Règle 1 :] En entrant dans le débat, vous êtes prêts à considérer sans violence tous les arguments.
 \item[Règle 2 :] Vous cherchez à connaître clairement vos propres opinions, leurs arguments et leurs limites, pour rester humble et être prêt à les voir contredites et modifiés.
 \item[Règle 3 :] Vous vous efforcez de tendre vers l'objectivité dans l'analyse de vos propres arguments et de ceux des autres.
 \item[Règle 4 :] De part sa vocation et son éthique, le club de réflexion citoyenne ne peut tolérer la violence sous toute ses formes, vous vous engagez à la dénoncer et à garder un certain détachement vis à vis de vos émotions pour qu'elles ne vous perturbent pas et que vous restiez maître de ce que vous dites.
\end{description}
Pour vous guider dans l'application de ces rêgles, elles sont appuyés par les documents en annexe, dont la lecture est conseillée, car ils fournissent un point de départ et des ouvertures à qui veut se former à l'art de l'argumentation et de la contre-argumentation. 

\end{document}